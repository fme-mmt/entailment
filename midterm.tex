\documentclass{beamer}

% There are many different themes available for Beamer. A comprehensive
% list with examples is given here:
% http://deic.uab.es/~iblanes/beamer_gallery/index_by_theme.html
% You can uncomment the themes below if you would like to use a different
% one:
%\usetheme{AnnArbor}
%\usetheme{Antibes}
%\usetheme{Bergen}
%\usetheme{Berkeley}
%\usetheme{Berlin}
%\usetheme{Boadilla}
%\usetheme{boxes}
%\usetheme{CambridgeUS}
%\usetheme{Copenhagen}
%\usetheme{Darmstadt}
%\usetheme{default}
%\usetheme{Frankfurt}
%\usetheme{Goettingen}
%\usetheme{Hannover}
%\usetheme{Ilmenau}
%\usetheme{JuanLesPins}
%\usetheme{Luebeck}
\usetheme{Madrid}
%\usetheme{Malmoe}
%\usetheme{Marburg}
%\usetheme{Montpellier}
%\usetheme{PaloAlto}
%\usetheme{Pittsburgh}
%\usetheme{Rochester}
%\usetheme{Singapore}
%\usetheme{Szeged}
%\usetheme{Warsaw}
\usepackage{color}
\usepackage{graphicx}
\usepackage{float}
\title{Recognizing Textual Entailment}

% A subtitle is optional and this may be deleted
\subtitle{MMT Project}

\author{ Nora Jarque \and V\'ictor Cristino \and Andreu Masdeu \and Xavier Salvador \and Dami\`a Fulton \and Raul P\'erez \and Gerard S\'anchez    \and Alejandro S\'aez}
% - Give the names in the same order as the appear in the paper.
% - Use the \inst{?} command only if the authors have different
%   affiliation.


\date{Midterm presentation, October 2017}


\AtBeginSubsection[]
{
  \begin{frame}<beamer>{Outline}
    \tableofcontents[currentsection,currentsubsection]
  \end{frame}
}

\begin{document}

\begin{frame}
  \titlepage
\end{frame}

\begin{frame}{Outline}
  \tableofcontents
  % You might wish to add the option [pausesections]
\end{frame}

% Section and subsections will appear in the presentation overview
% and table of contents.
\section{Description of the problem}


\begin{frame}{Description of the problem}        \centering    
    
    \begin{tabular}{c}
        P: A cat ate all the cookies. \\
        H: An animal ate all the cookies.\\
    \end{tabular}
    
    \bigskip
    
    \color{gray}
    
    \begin{tabular}{c}
        P: A cat ate all the cookies. \\
        H: A dog ate some of the cookies.\\
    \end{tabular}
    
    \bigskip
    
    \begin{tabular}{c}
        P: A cat ate some of the cookies. \\
        H: A cat ate all the cookies.\\
    \end{tabular}
\end{frame}

\begin{frame}{Description of the problem}
  \centering
  
  \color{gray}

  \begin{tabular}{c}
        P: A cat ate all the cookies. \\
        H: An animal ate all the cookies.\\
    \end{tabular}
    
    \bigskip
    
    \color{black}
    \begin{tabular}{c}
        P: A cat ate all the cookies. \\
        H: A dog ate some of the cookies.\\
    \end{tabular}
    
    \bigskip
    
    \color{gray}
    
    \begin{tabular}{c}
        P: A cat ate some of the cookies. \\
        H: A cat ate all the cookies.\\
    \end{tabular}
\end{frame}

\begin{frame}{Description of the problem}
    \centering
  \color{gray}
    \begin{tabular}{c}
        P: A cat ate all the cookies. \\
        H: An animal ate all the cookies.\\
    \end{tabular}
    
    \bigskip
    
    \begin{tabular}{c}
        P: A cat ate all the cookies. \\
        H: A dog ate some of the cookies.\\
    \end{tabular}
    \color{black}
    
    \bigskip
    
    \begin{tabular}{c}
        P: A cat ate some of the cookies. \\
        H: A cat ate all the cookies.\\
        \end{tabular}
\end{frame}

\begin{frame}{Description of the problem}
  \begin{center}
      OBJECTIVE:
  Automatize Recognizing Textual Entailment
\end{center}
\end{frame}


\section {Machine learning approach}
\begin{frame}{Machine learning approach}
  \begin{center}
  \includegraphics[width=0.5\textwidth]{lstm.png}
  \end{center}
  \begin{itemize}
  \item { +: More accuracy, parameter tuning is automatic, scalability is not constrained
  }
  \item { -: Bounds on uncertainty, tones of parameters, no explainability
  }
  \end{itemize}
\end{frame}


\section {Our approach}

\begin{frame}{ Our Approach}
  \begin{center}
  \includegraphics[width=1\textwidth]{mmt.png}
  \end{center}
\end{frame}

\section{Syntactic treatment}
\begin{frame}{Syntactic treatment: what is it?}
    \begin{itemize}
        \item Set of rules or principles that govern how words are put together to form phrases, well formed sequences of words. \break
        \item These rules allow us to organize the phrase in a rooted tree (digraph). \break
        \item Stanford CoreNLP: webpage that does this syntactic analysis, indicating in the edges the dependence of the nodes with his root.\break
    \end{itemize}
    \centering
    \includegraphics[width=0.5\textwidth]{CoreNLP.png}
\end{frame}

\begin{frame}{Syntactic treatment: what have we done?}
    \begin{itemize}
        \item Natural Language Processing (NLP) $\rightarrow$ aritificial intelligence \break
        \item Python $\rightarrow$ high-level programming language \break
        \item Spacy $\rightarrow$ syntactic parser, method within natural language processing\break
        \item NLTK $\rightarrow$ a suite of libraries and programs. We use Tree model \break
        \item Graphviz $\rightarrow$ program to show trees \break
        \item Jupyter Notebook, IPython $\rightarrow$ applications that allow editing and running notebook documents
    \end{itemize}
\end{frame}

\begin{frame}{Syntactic treatment: what have we got?}
    Two Python programs:\break
    \begin{itemize}
        \item \textbf{syntactic.py}: python program that, given some sentences, returns a vector of trees and show us with formal print and pretty print.\break
        \item \textbf{view.py}: python program that, given a tree, returns an .png image using graphviz.
    \end{itemize}
\end{frame}

\begin{frame}{Syntactic treatment: example}

    \includegraphics[width=0.9\textwidth]{syntactic_graphviz.png}
    %\includegraphics[width=0.5\textwidth]{Syntactic.png}
    %\includegraphics[width=0.4\columnwidth]{graphviz1.png}
    %\includegraphics[width=0.35\textwidth]{graphviz2.png}
\end{frame}


\begin{frame}{Modal Logic}
\begin{center}
    \begin{tabular}{  l  l  l }
    
    \textbf{Logic} & \textbf{Symbol} & \textbf{Expression} \\
    Modal & \fbox{$\phantom{}$} & It is necessary that...   \\ 
     & \diamond & It is possible that... \\ 
    Deontic & $O$ & It is obligatory that.. \\
     & $P$ & It is permitted that.. \\ 
      & $F$ & It is forbidden that.. \\ 
    Doxastic & $Bx$ & x believes that... \\ 
      & $Kx$ & x knows that... \\ 
    \end{tabular}
\end{center}
\end{frame}

\section{Semantic processing}
\begin{frame}{Semantic processing}
  \begin{itemize}
  \item {
    Using the syntactic tree, we want to obtain a logic expression for the phrase.
  }
  \item {
    We can achieve that with lambda calculus.
  }
  \end{itemize}
\end{frame}


\begin{frame}{Lambda calculus: Example}
\begin{center}
    Every boxer walks
\end{center}
  \begin{itemize}
      \item {
      every: $\lambda P.\lambda Q.\forall x(P@x\rightarrow Q@x)$
      }
      \item {
      boxer: $\lambda y.BOXER(y)$
      }
      \item {
      walks: $\lambda x.WALK(x)$
      }
  \end{itemize}
\end{frame}

\begin{frame}{Lambda calculus: Example}
  \begin{center}
      Every boxer
  \end{center}
  \begin{itemize}
      \item {
      $\lambda P.\lambda Q.\forall x(P@x\rightarrow Q@x)@\lambda y.BOXER(y)$
      }
      \item {
       $\lambda Q.\forall x(\lambda y.BOXER(y)@x\rightarrow Q@x)$
      }
      \item {
       $\lambda Q.\forall x(BOXER(x)\rightarrow Q@x)$
      }
      
  \end{itemize}

\end{frame}

\begin{frame}{Lambda calculus: Example}
  \begin{center}
  \includegraphics[width=1\textwidth]{lambdaex1.PNG}
  \end{center}
\end{frame}


\section{Conversion into Kanren}


\begin{frame}{Conversion into Kanren - Operations}
\color{black}
    \begin{center}
        x = var()\\
        run(1, x, eq(x, 5))\\
\color{red}
        $>>>$ 5
    \end{center}
\color{gray}
    \begin{center}
        z = var()\\
        run(1, x, eq(z, x), eq(z, 3))\\
        $>>>$ 3
    \end{center}
\end{frame}

\begin{frame}{Conversion into Kanren - Operations}
\color{gray}
    \begin{center}
        x = var()\\
        run(1, x, eq(x, 5))\\
\color{gray}
        $>>>$ 5
    \end{center}
\color{black}
    \begin{center}
        z = var()\\
        run(1, x, eq(z, x), eq(z, 3))\\
        \color{red}
        $>>>$ 3
    \end{center}
\end{frame}

\begin{frame}{Conversion into Kanren - Operations}
    \begin{center}
        run(0, x, membero(x, (1, 2, 3)),\\ membero(x, (2, 3, 4)))\\
        \color{red} $>>>$ (2, 3)
    \end{center}
    \color{gray}
    \begin{center}
        run(0, x, conde(\\
        membero(x, (1, 2, 3)),\\ 
        membero(x, (2, 3, 4)))\\
        )\\
        $>>>$ (1, 2, 3, 4)
    \end{center}
    \color{gray}
    \begin{center}
        run(0, x, conde(\\  
        eq(x, 1),\\   
        membero(x, (5, 6, 7)))
        )\\  
        $>>>$ (1, 5, 6, 7)
    \end{center}
\end{frame}

\begin{frame}{Conversion into Kanren - Operations}
    \color{gray}
    \begin{center}
        run(0, x, membero(x, (1, 2, 3)),\\ membero(x, (2, 3, 4)))\\
        $>>>$ (2, 3)
    \end{center}
    \color{black}
    \begin{center}
        run(0, x, conde(\\
        membero(x, (1, 2, 3)),\\ 
        membero(x, (2, 3, 4)))\\
        )\\
        \color{red}$>>>$ (1, 2, 3, 4)
    \end{center}
    \color{gray}
    \begin{center}
        run(0, x, conde(\\
        eq(x, 1),\\ 
        membero(x, (5, 6, 7)))
        )\\
        $>>>$ (1, 5, 6, 7)
    \end{center}
\end{frame}

\begin{frame}{Conversion into Kanren - Operations}
    \color{gray}
    \begin{center}
        run(0, x, membero(x, (1, 2, 3)),\\ membero(x, (2, 3, 4)))\\
        $>>>$ (2, 3)
    \end{center}

    \begin{center}
        run(0, x, conde(\\
        membero(x, (1, 2, 3)),\\ 
        membero(x, (2, 3, 4)))\\
        )\\
        $>>>$ (1, 2, 3, 4)
    \end{center}
    \color{black}
    \begin{center}
        run(0, x, conde(\\
        eq(x, 1),\\ 
        membero(x, (5, 6, 7)))
        )\\
        \color{red}$>>>$ (1, 5, 6, 7)
    \end{center}
\end{frame}

\begin{frame}{Conversion into Kanren - Operations}
\begin{center}
    def function(\color{red}arguments\color{black}):\\
    return conde(\color{gray}expression\color{black})
\end{center}

\begin{center}
run(0,\color{red}arguments\color{black}, function)
\end{center}
\end{frame}

\begin{frame}{Conversion into Kanren - Simpsons}
\begin{center}
    parent = \color{red}Relation()\color{black}\\
    facts (parent, ("Homer", "Bart"),\\
    ("Abe", "Homer"))
\end{center}
\begin{center}
    \color{orange}def \color{black}grandparent(u, v):\\
    return conde((parent(u, y), parent(y, v)))
\end{center}



\begin{center}
    run(0, x, grandparent(x, "Bart"))
\end{center}

\end{frame}



\begin{frame}{Conversion into Kanren - Simpsons}
  \begin{itemize}
    \item{
    Define relations: parent, married, grandparent, siblings.
    }
    \item{
    Connect the relations:
      \includegraphics[width=1\textwidth]{fs.PNG}
    }
  \end{itemize}
\end{frame}

\begin{frame}{Conversion into Kanren - Simpsons}
  \begin{itemize}
    \item{
    Create a function that adds facts to the database.
    }
    \item{
    Add some facts  
      \includegraphics[width=1\textwidth]{adddb.PNG}
    }
    \item{
    Ask some questions  
      \includegraphics[width=1\textwidth]{prints.PNG}
    }
    \item{
    Output:  
      \includegraphics[width=1\textwidth]{output.PNG}
    }
  \end{itemize}
\end{frame}

\section{Conclusion and goals looking forward}

\begin{frame}{Conclusion and goals looking forward}
  \begin{itemize}
  \item {
    Find a way to express negation in kanren.
  }
    \item {
    Add a dictionary of synonyms and inclusion relations in kanren.
  }
     \item {
    Do the syntactic, semantic and logic treatment of the sentences in the same program.
  }

  \end{itemize}
\end{frame}

% You can reveal the parts of a slide one at a time
% with the \pause command:
% All of the following is optional and typically not needed. 
\appendix
\section<presentation>*{\appendixname}
\subsection<presentation>*{For Further Reading}

\end{document}


